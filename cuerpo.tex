Las proteínas son macromoléculas formadas por cadenas de aminoácidos que desempeñan funciones esenciales en todos los sistemas biológicos. Están presentes en organismos tan diversos como procariostas, eucariotas, virus e incluso priones. Cada organismo contiene un gran conjunto de diferentes y específicas proteínas, en el caso de los humanos se estima en decenas de miles.\cite{Cozzone2010}

Estas moléculas participan en la mayoría de procesos biológicos, cumpliendo papeles claves a nivel estructural, funcional y de regulación. \cite{Ritchie2008}

La versatilidad de las proteínas se manifiesta en prácticamente todos los aspectos. La mayoría de las reacciones químicas en sistemas biológicos son catalizadas por proteínas complejas denominadas enzimas. Otras muchas proteínas son capaces de transportar y almacenar de iones, pequeñas moléculas o incluso electrones, este es el caso de la hemoglobina y el transporte de oxígeno en la sangre. Las proteínas también participan en el sistema inmunológico, como es el caso de los anticuerpos. También participan en la regulación de la expresión genética o en la diferenciación celular, entre otras muchas funciones \cite{Cozzone2010} \cite{Ritchie2008}.


En muchos de los caso estas proteínas no realizan sus funciones de forma individual, forman complejos interactuando con otras moléculas o proteínas \cite{Ritchie2008}. Estas asociaciones entre proteínas son específicas e implican números interacciones específicas que van a depender tanto de la estructura tridimensional como propiedades fisicoquímicas. /cite{Shoichet1991}. Los complejos formados pueden ser transitorios como es el caso de interacciones enzimáticas o estables como es el caso de los ribosomas \cite{Cozzone2010}.


%%%%%%%%%%%%%%%%%%% TECNICAS DE DETERMINACION DE LA ESTRUCTURA (proteinas y complejos)

%%%%%%%%%%%%%%%%%%% UTILIDAD DE DETERMINAR LA ESTRUCTURA (Farmacología)

%importancia de las proteínas en la biología y sus interacciones

La determinación de la estructura tridimensional de las proteínas y complejos es un objetivo clave de la biología, por sus implicaciones
en el entendimiento de las funciones e interacciones de estas macromoléculas. 
A pesar del desarrollo de las técnicas de determinación de estructuras de proteínas en los últimos años, son pocos los complejos
proteicos cuya estructura está determinada empíricamente. Es en este contexto donde los métodos computacionales
adquieren relevancia para la determinación de estas estructuras.
% estudio de las interacciones y estructuras de las proteínas y limitaciones de las técnicas experimentales

Uno de los enfoques de este ámbito es el docking de proteínas, se centra en predecir de la unión entre proteínas y ligandos
en función de la estructura de estas moléculas. Mediante este enfoque es posible predecir la estructura tridimensional de
estos complejos proteicos. 
El concepto de docking fue introducido por Wodak y Janin \cite{Janin2003} entre proteínas, para más tarde ser extendido a las interacciones con ligandos.
La unión con estas moléculas más pequeñas y sencillas es considerablemente más asequible, aunque tiene un coste computacional elevado.
El docking entre proteínas y un ligando se ha convertido en  áreas de investigación importante en el descubrimiento computacional de fármacos \cite{DeRuyck}, 
 ya que la función de las
proteínas puede modificarse mediante la unión de ligandos específicos permitiendo el diseño de fármacos dirigidos.

\begin{comment}
    %uso farmacologico del docking y ejemplos 
Sequencing of the human genome has led to an increase in the number of new
therapeutic targets for pharmaceutical research. In addition, high-throughput crystallography and nuclear magnetic resonance methods have been further developed and
contributed to the acquisition of the atomic structures of proteins and protein–ligand
complexes of an increasing level of detail.1
 When the three-dimensional structure of
the target, even from experiments or computing, exists, a frequently used technique to
design inhibitor molecules is structure-based drug design (SBDD)

Ligand binding is the key step in enzymatic reactions and,
thus, for their inhibition. Therefore, a detailed understanding
of interactions between small molecules and proteins may
form the basis for a rational drug design strategy.45–48 This
approach was widely considered in order to design molecules
addressing a broad range of major pathologies such as cancers49,50 or cardiovascular diseases.5

\end{comment}

El docking de proteínas combina física, química, biología y matemáticas con informática. Los métodos tradicionales se
centran en el uso de funciones de puntuación basadas en principios físicos y algoritmos de búsqueda. Estos métodos son
computacionalmente muy costosos. \cite{??}




\begin{comment}
Protein–protein docking actually predates protein–ligand
(small molecule) docking, as the concept of protein docking
introduced by Wodak and Janin80 was later extended to the
interaction between macromolecules and small ligands.81 The
treatment of flexibility in the binding process is considerably
easier with small molecules, even though a considerable
computational cost is involved, and small molecule docking has become one of the most active research areas in
computational drug discovery. 
\end{comment}



En el último tiempo han ganado importancia algoritmos basados en deep learning. Estos métodos proporcionan menores
tiempos de ejecución, aunque en muchos casos su rendimiento no ha superado los métodos tradicionales. Uno de los enfoques
basados en aprendizaje profundo es DiffDock , un modelo generativo basado en difusión mediante el uso de redes neuronales
gráficas.
% usando deeplearning

En 2024 el Premio Nobel de Química ha sido otorgado a AlphaFold, un modelo basado en redes neuronales capaz de predecir
la estructura de proteínas con una precisión comparable a los métodos experimentales según sus autores. El modelo predice
las coordenadas tridimensionales de los átomos pesados en base a la secuencia de aminoácidos. La versión más reciente de
este modelo, AlphaFold 3, introduce una arquitectura basada en difusión que permite predecir también la estructura
complejos proteicos, permitiendo así el estudio del acoplamiento entre proteínas y ligandos.


%evaluación del docking añadir más cosas y objetivo principal del trabajo
En este trabajo se plantea el uso de nuevos métodos de deep learning para la realización de acoplamiento de proteinas además
de una posterior evaluación de los ensamblajes obtenidos con herramientas como PoseBusters.
